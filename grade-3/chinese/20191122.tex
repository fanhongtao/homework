\documentclass[UTF8]{ctexart}
\usepackage{amsmath}
\usepackage{geometry}
\usepackage{CJKfntef}  % 文字下加点
\usepackage{tabularx}

\geometry{a4paper,left=2cm,right=2cm,top=2cm,bottom=2cm}
\linespread{1.5}

\usepackage{../../packages/picins/picins}

% 绘制带圈的数字
\usepackage{tikz}
\usepackage{etoolbox}
\newcommand{\circled}[2][]{\tikz[baseline=(char.base)]
    {\node[shape = circle, draw, inner sep = 1pt]
    (char) {\phantom{\ifblank{#1}{#2}{#1}}};%
    \node at (char.center) {\makebox[0pt][c]{#2}};}}
\robustify{\circled}

% 题目相关的配置
\newcounter{itemdati}[section]          % 大题的计数器: 一、二、三、……
\newcounter{itemzhongti}[itemdati]      % 大题的计数器:(一)(二)(三)……
\newcounter{itemxiaoti}[itemzhongti]    % 小题的计数器: 1. 2. 3. ……
\newcounter{itemxiaoxiaoti}[itemxiaoti] % 小小题的计数器: (1) (2) (3) ……
\newcounter{itemweiti}[itemxiaoxiaoti]  % 微题的计数器: ①  ②  ③ 	 ……
\newcommand{\dati}[1]{ % 大题的标题
  \stepcounter{itemdati}
  \hspace{-2.8em}\textbf{\chinese{itemdati}、 #1}
}
\newcommand{\zhongti}[1]{ % 中题的标题
  \stepcounter{itemzhongti}
  \hspace{-2.8em}(\chinese{itemzhongti}) #1
}
\newcommand{\xiaoti}[1]{ % 小题的标题
  \stepcounter{itemxiaoti}
  \hangafter 1\setlength{\hangindent}{3.2em}{\arabic{itemxiaoti}. #1}
}
\newcommand{\xiaoxiaoti}[1]{ % 小小题的标题
  \stepcounter{itemxiaoxiaoti}
  \hangafter 1\setlength{\hangindent}{5.4em}{\hspace{1em}(\arabic{itemxiaoxiaoti})#1}
}
\newcommand{\weiti}[1]{ % 微题的标题
  \stepcounter{itemweiti}
  \hangafter 1\setlength{\hangindent}{5.4em}{\hspace{3em}\circled{\arabic{itemweiti}}#1}
}

\newcommand{\xhx}[1][6em] {\underline{\hspace{#1}} } % 下划线

\newcommand{\kh}[1][2em] {\mbox{(\hspace{#1})}} % 括号

\newcommand{\kbai}[1][1em] {\hspace{#1}} % 空白

% 带括号:指题目后面,有一个向右对齐的括号。形如:
%    题目标题  。。。。 (   )
% 常用于 判断对错 或 填写选择项
\newcommand{\daikuohao}[1]{#1 \hfill \kh}

%---- 将若干项目放置在一行 -----
% 这一组命令的 第1个参数(可选),用于控制每个项目所占的宽度
\newcommand{\twoInLine}  [3][10em] {\begin{tabular}[t]{*{2}{@{}p{#1}}} #2 & #3\end{tabular}}
\newcommand{\threeInLine}[4][10em] {\begin{tabular}[t]{*{3}{@{}p{#1}}} #2 & #3 & #4\end{tabular}}
\newcommand{\fourInLine} [5][10em] {\begin{tabular}[t]{*{4}{@{}p{#1}}} #2 & #3 & #4 & #5\end{tabular}}
\newcommand{\fiveInLine} [6][6.8em]{\begin{tabular}[t]{*{5}{@{}p{#1}}} #2 & #3 & #4 & #5 & #6\end{tabular}}
\newcommand{\sixInLine}  [7][5.8em]{\begin{tabular}[t]{*{6}{@{}p{#1}}} #2 & #3 & #4 & #5 & #6 & #7\end{tabular}}

%--- 将若干项目放置为一列(前加表示分组的大括号) ---
\newcommand{\twoInRow}  [2] {$\Bigg\{ \begin{tabular}{@{}l@{}} \text{#1}\\ \text{#2} \end{tabular} $ }
\newcommand{\threeInRow}[3] {$\Bigg\{ \begin{tabular}{@{}l@{}} \text{#1}\\ \text{#2}\\ \text{#3} \end{tabular} $ }
\newcommand{\fourInRow} [4] {$\Bigg\{ \begin{tabular}{@{}l@{}} \text{#1}\\ \text{#2}\\ \text{#3}\\ \text{#4} \end{tabular} $ }

%---- 选择题选项 ----
% 四个选项排成一行
\newcommand{\fourch}[4]{\begin{tabular}{*{4}{@{}p{3.5cm}}}(A)~#1 & (B)~#2 & (C)~#3 & (D)~#4\end{tabular}}
% 每两个选项排成一行,共两行
\newcommand{\twoch}[4]{\begin{tabular}{*{2}{@{}p{7cm}}}(A)~#1 & (B)~#2\end{tabular}\\\begin{tabular}{*{2}{@{}p{7cm}}}(C)~#3 &(D)~#4\end{tabular}}
% 每个选项单独占一行,共四行
\newcommand{\onech}[4]{(A)~#1 \\ (B)~#2  \\ (C)~#3  \\ (D)~#4}  % 一行


% 设置页脚
\usepackage{fancyhdr}
\pagestyle{fancy}
\renewcommand\headrulewidth{0pt}
\fancyfoot[C]{三年级(上) 语文(R) \kbai 13 — \thepage}


% 自定义命令
\newcommand{\pailie}[3]{
\begin{math}
  \begin{cases}
    \text{#1} \\
    \text{#2} \\
    \text{#3}
  \end{cases}
\end{math}
}

\newcommand{\khao}{\kh[1.5cm]}

\newcommand{\sanxuanyi}[3]{\begin{tabular}{*{3}{p{8em}}p{6em}l}
  A. #1 & B. #2 & C. #3 & & \kh
\end{tabular}}

% 四个一行
\newcommand{\sigeyihang}[4]{\fourInLine{#1}{#2}{#3}{#4}}  % 为了减少代码修改,保留这个命令的定义。

\begin{document}

\begin{center}
\begin{LARGE}
\textbf{三年级(上) 语文(R)}
\end{LARGE}

\begin{Large}
期末专项复习卷(一)

拼音、字词
\end{Large}

班 级\xhx \kbai 姓 名\xhx
\end{center}

\dati{看拼音,写字词。}

\zhongti{基础题精选。}

\xiaoti{看拼音,写同音字。}

\begin{tabular}{c c c c c}
jī & & liú & & jìng \\
\pailie{\kh 动}{打 \kh}{\kh 场} & \kbai[2cm] 
& \pailie{停 \kh}{\kh 海}{\kh 水} & \kbai[2cm] 
& \pailie{\kh 爱}{石 \kh}{安 \kh} \\
\end{tabular}


\xiaoti{看拼音,写词语。}

% \begin{tabular}{*{5}{p{8em}}}
\begin{tabularx}{46em}%
{*{5}{>{\centering\arraybackslash}X}}
fú zhuāng & guó qí & lǎng dú & zhāo yǐn & yǐng zi \\
\khao &  \khao & \khao & \khao & \khao \\

huāng yě & hù xiāng & zì rán & jìn tóu & gē chàng \\
\khao &  \khao & \khao & \khao & \khao \\

xiān zǐ & qì wèi & wéi qún & nuǎn huo & kě lián \\
\khao &  \khao & \khao & \khao & \khao \\

hán lěng & yào hǎo & dāng rán & yǎn lèi & gāng cái \\
\khao &  \khao & \khao & \khao & \khao \\

guāng liàng & mǔ qīn & gēn běn & zhī zhū & yīn cǐ \\
\khao &  \khao & \khao & \khao & \khao \\

bǔ yú & chuāng qián & guān chá & hé lǒng & shuì jiào \\
\khao &  \khao & \khao & \khao & \khao \\

yǒu qù & gǎn jǐn & yóu dòng & jiē dào & kào àn \\
\khao &  \khao & \khao & \khao & \khao \\

yào cái & yǔ dī & suǒ yǒu & qīng kuài & máng rán \\
\khao &  \khao & \khao & \khao & \khao \\

zǒng shì & lù shuǐ & hū rán & zhù shì & zhǎng shēng \\
\khao &  \khao & \khao & \khao & \khao \\

chí jiǔ & lèi shuǐ & píng xī & miàn duì & kěn dìng \\
\khao &  \khao & \khao & \khao & \khao \\
\end{tabularx}

\zhongti{较难题精选。}

\xiaoti{看拼音,写词语。}

冬天,虽然天气 yán hán \khao ,却是人们 xǐ ài \khao 的冬节,雪花 fēi wǔ \khao ,
大人、小孩尽情 wán shuǎ \khao ,大街小巷,人人 gē chàng \khao ,处处 kuáng huān \khao ,
许多人的 tóng nián \khao 因为冬天而 gèng jiā \khao 绚丽多彩!

\newpage % ----------------------------------------------

{\linespread{2}}

\xiaoti{我会拼,我会写。}

新年到了,同学们穿上 xiān yàn \khao 的服装,参加晚会,小乐手们 yǎn zòu \khao 着乐曲,摇晃着 nǎo dai \khao ,可投入了!
舞蹈队的“小黄鹂”们扇动“chì bǎng \khao”,边跳边唱……观众们都很 xǐ huan \khao 这些节目,会场里响起了 rè liè \khao 的掌声。

\xiaoti{看拼音,写词语。}

同学们一起去 dēng shān \khao 。到了半山腰,很多人累得想 fàng qì \khao 了,只有一部分人 jiān chí \khao 爬到了峰顶。

雷公公 cōng cōng \khao 地来了,天空顿时 àn \kh[1cm] 了下来,他 shēn shǒu \khao 抓来几朵乌云,用力一挤,yì kē kē \khao 雨珠滴落下来,前赶后 zhuī \kh[1cm],汇成小溪,流进 gōu \kh[1cm] 渠。

妈妈 róu ruǎn \khao 的手指轻轻地 qiāo jī \khao 着琴键。

\dati{字音、字形。}

\zhongti{基础题精选。}

\xiaoti{给加点字选择正确的读音,打上“$\surd$”。}

\xiaoxiaoti{
  \hspace{-1em}\begin{tabular}[t]{*{3}{p{12em}}} % tabular 会多一个空格,将它反向删除
    装\CJKunderdot{载} (zǎi \kbai zài)   & 海\CJKunderdot{参} (cān \kbai shēn) &  胳\CJKunderdot{臂} (bì \kbai bei)\\
    \CJKunderdot{凌} (líng \kbai lín)乱  & 根\CJKunderdot{茎} (jīn \kbai jīng) &  玫\CJKunderdot{瑰} (guī \kbai guì)
  \end{tabular}
}

\xiaoxiaoti{
  \hspace{-1em}\begin{tabular}[t]{*{3}{p{12em}}}
    蟋\CJKunderdot{蟀}(suài \kbai shuài) & \CJKunderdot{材} (cái \kbai chái)料 & \CJKunderdot{搭} (dā \kbai tǎ)船 \\
    \CJKunderdot{沾}(zhān \kbai zhàn)水 & \CJKunderdot{钓} (diào \kbai gōu)鱼 & \CJKunderdot{增} (zēn \kbai zēng)加
  \end{tabular}
}

\xiaoti{给加点字选择正确的读音或字,打上“$\surd$”。}

\hspace{2em}\CJKunderdot{橙}(chéng \kbai chén)黄 \kbai[3em] \CJKunderdot{靴}(xiē \kbai xuē)子 \kbai[3em]  基\CJKunderdot{础}(chǔ \kbai cǔ)

\hspace{2em}(迅 \kbai 讯)速 \kbai[3em] 热(烈 \kbai 列) \kbai[3em] (交 \kbai 郊)外 \kbai[3em] 合(龙 \kbai 拢)

\xiaoti{给加点字选择正确的读音,打上“$\surd$”。}

\xiaoxiaoti{请你仔细看看,茶\CJKunderdot{几}(jǐ  \kbai jī)上摆放着\CJKunderdot{几}(jǐ  \kbai jī)种图形。}

\xiaoxiaoti{爸爸依依不\CJKunderdot{舍}(shě  \kbai shè)地离开了我的宿\CJKunderdot{舍}(shě  \kbai shè)。}

\xiaoti{给加点字选择正确的读音,打上“$\surd$”。}

\xiaoxiaoti{夏天,每\CJKunderdot{逢}(féng  \kbai fén)爷爷、奶奶等一些长辈来我家庭院外的大树下乘凉时,我都会搬来一些长\CJKunderdot{凳}(dèng  \kbai dèn)让他们坐。}

\xiaoxiaoti{站在\CJKunderdot{中}(zhōng  \kbai zhòng)间的战士不幸\CJKunderdot{中}(zhōng  \kbai zhòng)弹了,躺在地上。}

\xiaoti{我会选择正确的读音打上“$\surd$”。}

\xiaoxiaoti{寒\CJKunderdot{假}(jiǎ  \kbai jià)里,工商部门的叔叔将开展打\CJKunderdot{假}(jiǎ  \kbai jià)活动。}

\xiaoxiaoti{答\CJKunderdot{应}(yīng  \kbai yìng)了别人的事,就应\CJKunderdot{应}(yīng  \kbai yìng)该认真去完成。}

\xiaoxiaoti{山上\CJKunderdot{落}(lào  \kbai luò)下的石头砸中了王大爷的腿,使他\CJKunderdot{落}(lào  \kbai luò)下了残疾。}

\newpage

\xiaoti {选择题。(填序号)}

\xiaoxiaoti{下面每小题中都有一个带点字的注音是错的,把它选出来。}

\weiti{\sanxuanyi {答\CJKunderdot{应}(yīng)} {\CJKunderdot{呢}(ní)喃} {\CJKunderdot{挑}(tiǎo)灯} }

\weiti{\sanxuanyi {中\CJKunderdot{弹}(tán)} {羊\CJKunderdot{圈}(juàn)} {胳\CJKunderdot{臂}(bei)} }

\weiti{\sanxuanyi {姿\CJKunderdot{势}(shì)} {增\CJKunderdot{添}(tiān)} {\CJKunderdot{鹦}(yīn)鹉} }

\xiaoxiaoti{下面每小题中都有一个词语含有错别字,把它选出来。}

\weiti{\sanxuanyi{粗状}{演奏}{归拢}}

\weiti{\sanxuanyi{墙壁}{排烈}{苍翠}}

\weiti{\sanxuanyi{温柔}{满栽}{玩耍}}

\xiaoti{选字填空。}

\xiaoxiaoti{
  \hspace{-1em}\begin{tabular}[t]{*{5}{p{6em}}}
    【载 \kbai 栽】 & \kh 树 & 装 \kh   & \kh 种   & \kh 客 \\
    【鱼 \kbai 渔】 & \kh 虾 & \kh 民   & 鲨 \kh   & \kh 船 \\
    【乱 \kbai 刮】 & \kh 风 & \kh 糟糟 & \kh 胡子 & 杂\kh \\
  \end{tabular}
}

\xiaoxiaoti{
  \hspace{-1em}\begin{tabular}[t]{*{4}{p{6em}}}
    【击 \kbai 激】 &  【临 \kbai 林】 & 【想 \kbai 响】 & 【融 \kbai 容】\\
    \kh 动 & 树 \kh & \kh 亮 & \kh 易 \\
    打 \kh & \kh 时 & 思 \kh & \kh 合 \\
  \end{tabular}
}

\zhongti{较难、易错题精选。}

\xiaoti{\daikuohao{下列词语中加点字读音完全正确的一项是}}
\twoch{\CJKunderdot{挑}促织(tiāo) \kbai 海\CJKunderdot{参}(shēn)}{哈\CJKunderdot{欠}(qiè) \kbai \CJKunderdot{稍}息(shǎo)}
      {\CJKunderdot{落}下残疾(lào) \kbai \CJKunderdot{压}根(yà)}{胳\CJKunderdot{臂}(bì) \kbai 停\CJKunderdot{当}(dàng)}

\xiaoti{\daikuohao{下面加点字读音有错误的一项是}}
\twoch{手\CJKunderdot{臂}(bì) \kbai 黑\CJKunderdot{洞}(dòng)}{发\CJKunderdot{愁}(cóu) \kbai 点\CJKunderdot{燃}(lán)}
      {\CJKunderdot{盛}开(shèng) \kbai \CJKunderdot{罚}(fá)站}{照\CJKunderdot{例}(lì) \kbai \CJKunderdot{翅}膀(chì)}

\xiaoti{\daikuohao{下列词语中加点字读音全部正确的一项是} \\
  \onech{\sigeyihang{宿\CJKunderdot{舍}(shè)}{\CJKunderdot{落}下残疾(lào)}{\CJKunderdot{兴}奋(xīng)}{\CJKunderdot{凌}乱(líng)}}
    {\sigeyihang{\CJKunderdot{犹}豫(yóu)}{\CJKunderdot{扎}在背上(zā)}{\CJKunderdot{担}保(dān)}{胳\CJKunderdot{臂}(bei)}}
    {\sigeyihang{\CJKunderdot{稍}息(shào)}{一\CJKunderdot{骨}碌(gū)}{\CJKunderdot{背}诵(bèi)}{茶\CJKunderdot{几}(jǐ)}}
    {\sigeyihang{低\CJKunderdot{吟}(yín)}{\CJKunderdot{膝}盖(xī)}{\CJKunderdot{中}弹(zhòng)}{\CJKunderdot{处}罚(chǔ)}}
}

\xiaoti{\daikuohao{以下选项中,字音和字形都没有错误的一项是} \\
  \onech{\sigeyihang{或(huò)者}{鲫(jì)鱼}{玩耍}{耳闻目堵}}
    {\sigeyihang{富饶(ráo)}{禁(jìn)令}{饮料}{层林尽染}}
    {\sigeyihang{山洼(wā)}{鹦(yīng)鹉}{搭船}{五采缤纷}}
    {\sigeyihang{犹(yiú)豫}{玫瑰(guī)}{打猎}{各种各样}}
}

\newpage %-------------

\xiaoti{\daikuohao{下面成语中有错别字的一组是} \\
  \onech{\sigeyihang{摇头晃脑}{张牙舞爪}{提心吊胆}{眼疾手快}}
    {\sigeyihang{层林尽染}{春华秋实}{五谷丰登}{一叶知秋}}
    {\sigeyihang{窗明几净}{百衣百顺}{耳闻目睹}{一本正经}}
    {\sigeyihang{波澜壮阔}{寸步难行}{细嚼慢咽}{秋风习习}}
}

\dati{查字典填空。}

\zhongti{基础题精选。}

\xiaoti{“封”字用部首查字法先查 \xhx[3em] 部,再查 \xhx[3em] 画。“封”在字典里的解释有: \circled{1}封闭;\circled{2}疆界;\circled{3}量词,用于封起来的东西。\\
  \hspace*{1em} “密密层层的树叶把森林封得严严实实的”中“封”应取 \xhx 种解释。这个“封”形象地写出了 \xhx 。\\
  A. 树木太多,挡住了人们的道路 \kbai B. 树木枝叶繁茂的景象
}

\xiaoti{“爽”字用部首查字法,应先查 \xhx[3em] 部,再查 \xhx[3em] 画。它在字典里的解释有: \circled{1}明朗,清亮;\circled{2}(性格)率直;\circled{3}舒服,畅快;\circled{4}违背。选出下列词语中“爽”字的意思。(填序号)\\
  \sigeyihang{秋高气爽\kh}{毫厘不爽\kh}{精神爽朗\kh}{性格豪爽\kh}
}

\xiaoti{“疾”字用音序查字法应先查大写字母 \kh[3em] ,按部首查字法应查 \kh[3em] 部,再查 \kh[3em] 画。在字典中,“疾”字有四种解释: \circled{1}疾病;\circled{2}痛苦;\circled{3}痛恨;\circled{4}急速,猛烈。
  “残疾”的“疾”应选\kh[3em],“疾恶如仇”的“疾”应选\kh[3em],“疾风”的“疾”应选\kh[3em]。(填序号)
}

\xiaoti{“荒”字的部首是 \kh[3em] ,除去部首还有 \kh[3em] 画。“荒”的意思有: \circled{1}长满野草或无人耕种;\circled{2}荒凉,偏僻;\circled{3}不合情理,不正确。选出下列句子中“荒”字的意思。(填序号)\\
  \xiaoxiaoti{这里曾经是个\CJKunderdot{荒}村,近年因为发展旅游业变得十分热闹。\kh}\\
  \xiaoxiaoti{大家都劝马小虎赶快打消这个\CJKunderdot{荒}唐的念头。\kh}\\
  \xiaoxiaoti{这个村子有很多可以耕种的\CJKunderdot{荒}地。\kh}
}

\parpic[sr]{%
\includegraphics[width=6cm]%
{zhang.jpg}}

\zhongti{较难、易错题。}

\indent“张”字用部首查字法先查部首 \xhx[3em],再查 \xhx[3em] 画;用音序查字法先查大写字母 \xhx[3em],再查音节 \xhx[3em]。
“张”在字典里的意思有这种几种(如右图所示),给下列句子中的“张”字选择正确的意思。

% 下面几个小题,因为右侧有图片,所以通过加 hspace 强制不缩进。
\hspace{-2.5em}\xiaoti{蒲公英的花就像我们的手掌,可以\CJKunderdot{张}开、合上。\kh}

\hspace{-2.5em}\xiaoti{小明上课不认真听,一到考试就东\CJKunderdot{张}西望。\kh}

\hspace{-2.5em}\xiaoti{新年到了,大街上到处\CJKunderdot{张}灯结彩。\kh}

\hspace{-2.5em}\xiaoti{我家客厅的墙上挂着一\CJKunderdot{张}旧照片。\kh}

\end{document}
