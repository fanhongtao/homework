\documentclass[UTF8]{ctexart}
\usepackage{amsmath}
\usepackage{geometry}
\usepackage{CJKfntef}  % 文字下加点
\geometry{a4paper,left=2cm,right=2cm,top=2cm,bottom=2cm}
{\linespread{2.0}

% \pagestyle{plain}
% 绘制带圈的数字
\usepackage{tikz}
\usepackage{etoolbox}
\newcommand{\circled}[2][]{\tikz[baseline=(char.base)]
    {\node[shape = circle, draw, inner sep = 1pt]
    (char) {\phantom{\ifblank{#1}{#2}{#1}}};%
    \node at (char.center) {\makebox[0pt][c]{#2}};}}
\robustify{\circled}

% 题目相关的配置
\newcounter{itemdati}[section]          % 大题的计数器: 一、二、三、……
\newcounter{itemzhongti}[itemdati]      % 大题的计数器:(一)(二)(三)……
\newcounter{itemxiaoti}[itemzhongti]    % 小题的计数器: 1. 2. 3. ……
\newcounter{itemxiaoxiaoti}[itemxiaoti] % 小小题的计数器: (1) (2) (3) ……
\newcounter{itemweiti}[itemxiaoxiaoti]  % 微题的计数器: ①  ②  ③ 	 ……
\newcommand{\dati}[1]{ % 大题的标题
  \stepcounter{itemdati}
  \hspace{-2.8em}\textbf{\chinese{itemdati}、 #1}
}
\newcommand{\zhongti}[1]{ % 中题的标题
  \stepcounter{itemzhongti}
  \hspace{-2.8em}(\chinese{itemzhongti}) #1
}
\newcommand{\xiaoti}[1]{ % 小题的标题
  \stepcounter{itemxiaoti}
  \hangafter 1\setlength{\hangindent}{3.2em}{\arabic{itemxiaoti}. #1}
}
\newcommand{\xiaoxiaoti}[1]{ % 小小题的标题
  \stepcounter{itemxiaoxiaoti}
  \hangafter 1\setlength{\hangindent}{5.4em}{\hspace{1em}(\arabic{itemxiaoxiaoti})#1}
}
\newcommand{\weiti}[1]{ % 微题的标题
  \stepcounter{itemweiti}
  \hangafter 1\setlength{\hangindent}{5.4em}{\hspace{3em}\circled{\arabic{itemweiti}}#1}
}

\newcommand{\xhx}[1][6em] {\underline{\hspace{#1}}} % 下划线

\newcommand{\kh}[1][2em] {\mbox{(\hspace{#1})}} % 括号

\newcommand{\kbai}[1][1em] {\hspace{#1}} % 空白

% 选择题题目
\newcommand{\xuanzeti}[1]{#1 \hfill \kh}

%---- 将若干项目放置在一行 -----
% 这一组命令的 第1个参数(可选),用于控制每个项目所占的宽度
\newcommand{\twoInLine}  [3][10em] {\begin{tabular}[t]{*{2}{@{}p{#1}}} #2 & #3\end{tabular}}
\newcommand{\threeInLine}[4][10em] {\begin{tabular}[t]{*{3}{@{}p{#1}}} #2 & #3 & #4\end{tabular}}
\newcommand{\fourInLine} [5][10em] {\begin{tabular}[t]{*{4}{@{}p{#1}}} #2 & #3 & #4 & #5\end{tabular}}
\newcommand{\fiveInLine} [6][6.8em]{\begin{tabular}[t]{*{5}{@{}p{#1}}} #2 & #3 & #4 & #5 & #6\end{tabular}}
\newcommand{\sixInLine}  [7][5.8em]{\begin{tabular}[t]{*{6}{@{}p{#1}}} #2 & #3 & #4 & #5 & #6 & #7\end{tabular}}

%---- 选择题选项 ----
% 四个选项排成一行
\newcommand{\fourch}[4]{\begin{tabular}{*{4}{@{}p{3.5cm}}}(A)~#1 & (B)~#2 & (C)~#3 & (D)~#4\end{tabular}}
% 每两个选项排成一行,共两行
\newcommand{\twoch}[4]{\begin{tabular}{*{2}{@{}p{7cm}}}(A)~#1 & (B)~#2\end{tabular}\\\begin{tabular}{*{2}{@{}p{7cm}}}(C)~#3 &(D)~#4\end{tabular}}
% 每个选项单独占一行,共四行
\newcommand{\onech}[4]{(A)~#1 \\ (B)~#2  \\ (C)~#3  \\ (D)~#4}  % 一行


% 设置页脚
\usepackage{fancyhdr}
\pagestyle{fancy}
\renewcommand\headrulewidth{0pt}
\fancyfoot[C]{四年级(上) 语文(R) \kbai 14 — \thepage}

\begin{document}

\begin{center}
\begin{LARGE}
\textbf{四年级(上) 语文(R)}
\end{LARGE}

\begin{Large}
期末专项复习卷(二)

词 \kbai 语
\end{Large}

班 级\xhx \kbai 姓 名\xhx
\end{center}

\dati{基础题精选。}

\zhongti{按要求定词语。}

\xiaoti{ABB 式的词语: 暖洋洋  \xhx  \xhx \xhx}

\xiaoti{ABAC 式的词语: 人山人海  \xhx  \xhx \xhx}

\xiaoti{AABB 式的词语: 坑坑洼洼  \xhx  \xhx \xhx}

\xiaoti{AABC 式的同语: 愤愤不平  \xhx  \xhx \xhx}

\xiaoti{ABCC 式的词语: 白发苍苍  \xhx  \xhx \xhx}

\xiaoti{带有人体器官的词语: 手舞足蹈  \xhx  \xhx \xhx}

\xiaoti{带有数字的词语: 横七竖八  \xhx  \xhx \xhx}

\xiaoti{带有反义词的词语: 左顾右盼  \xhx  \xhx \xhx}

\zhongti{把下列词语补充完整。}

眉\kh目\kh  \kbai  \kh\kh玉立  \kbai  明\kh\kh齿 \kbai  文质\kh\kh

相貌\kh\kh  \kbai  威风\kh\kh  \kbai  \kh大\kh圆 \kbai  \kh\kh精悍

容\kh焕\kh  \kbai  \kh发童\kh  \kbai  \kh眉\kh目 \kbai  老态\kh\kh

\zhongti{照样子,在括号里填上合适的词语。}

\xiaoti{(优美)的风景  \kbai  \kh[1cm]的豆荚   \kbai  \kh[1cm]的牙齿}

\xiaoti{(剧烈)地抖动  \kbai  \kh[1cm]地移动   \kbai  \kh[1cm]地避开}

\xiaoti{(抄写)文章    \kbai  \kh[1cm]生活   \kbai   \kh[1cm]奇迹}

\zhongti{把下列词语中不是同一类的用 “\xhx[1cm]” 画出。}

\xiaoti{霎时 \kbai 顿时 \kbai 忽然 \kbai 小时 \kbai 刹那}

\xiaoti{玫瑰 \kbai 茉莉 \kbai 牡丹 \kbai 海棠 \kbai 鲜花}

\xiaoti{韭菜 \kbai 鲍鱼 \kbai 芹菜 \kbai 青蒜 \kbai 辣椒}

\xiaoti{提心吊胆 \kbai 欣喜若狂 \kbai 魂飞魄散 \kbai 胆战心惊}

\newpage

\zhongti{写出下列诗句中词语的意思。}

\xiaoti{可怜九月初三夜,露似真珠月似弓。}

\kbai 可怜:\xhx \kbai 真珠:\xhx

\xiaoti{梅雪争春未肯降,骚人阁笔费评章。}

\kbai降:\xhx \kbai 骚人:\xhx

\xiaoti{但使龙城飞将在,不教胡马度阴山。}

\kbai 但使:\xhx \kbai 教:\xhx 

\xiaoti{醉卧沙场君莫笑,古来征战几人回?}

\kbai 沙场:\xhx \kbai 回:\xhx

\xiaoti{生当作人杰,死亦为鬼雄。}

\kbai 人杰:\xhx \kbai 鬼雄:\xhx 

\zhongti{写出下列带点词的近义词。}

\xiaoti{\CJKunderdot{宽阔}的钱塘江横卧在眼前。 \kh[1cm] }

\xiaoti{浪潮越来越近,\CJKunderdot{犹如}千万匹白色战马齐头并进,浩浩荡荡地飞奔而来。\kh[1cm] }

\xiaoti{不要瞧不起那些灰色的脚,那些脚巴在墙上相当\CJKunderdot{牢固},要是你的手指不费一点儿劲,休想拉下爬山虎的一根茎。\kh[1cm] }

\xiaoti{蟋蟀钻到土底下干活,如果感到\CJKunderdot{疲劳},它就在未完工的家门口休息一会儿,头朝着外面,触须轻微地摆动。\kh[1cm] }

\xiaoti{在它看来,猎狗是个多么\CJKunderdot{庞大}的怪物啊!\kh[1cm] }

\dati{较难题精选。}

\zhongti{选出括号里用得恰当的词。}

\xiaoti{得知普罗米修斯从天上取走火种的消息,众神的领袖宙斯气急败坏,决定给普罗米修斯以最\kbai(严肃 \kbai 严厉)的惩罚,吩咐火神立即执行。}

\xiaoti{它嘴角嫩黄,头上长着绒毛,(分明 \kbai 证明)是刚出生不久,从巢里掉下来的。} 

\xiaoti{要不是你的勇气鼓舞我,我还下不了决心哩!现在(竟然 \kbai 居然)爬上来了!} 

\xiaoti{魏校长怎么也没想到,一个十二三岁的孩子,(竟然 \kbai 居然)有如此的抱负和胸怀!} 

\xiaoti{奇怪的是,我的陀螺个头小,却(顽强 \kbai 顽固)得出奇!} 

\xiaoti{是啊,要是我会豁虎跳·这场戏就不至于(砸破 \kbai 砸锅)了。} 

\newpage

\zhongti{选择恰当的关联词语填空}

\kbai 因为……所以…… \kbai 尽管……还…… \kbai 既……又……

\kbai 虽然……然而…… \kbai 如果……就…… \kbai 即使……也……

\kbai 不是……而是…… \kbai 无论……都…… \kbai 宁可……也……

\xiaoti{归巢的鸟儿,\kh[1cm] 是倦了,\kh[1cm] 驮着斜阳回去。}

\xiaoti{星光在我们的肉眼里 \kh[1cm] 微小,\kh[1cm] 它使我们们觉得光明无处不在。}

\xiaoti{这里 \kh[1cm] 温暖,\kh[1cm] 舒适;白天明亮,夜间黑喑。}

\xiaoti{\kh[1cm] 你能捉住我,\kh[1cm] 请来吧!}

\xiaoti{蝙蝠在夜里飞行,\kh[1cm] 一根极细的电线,它 \kh[1cm] 能灵巧地避开。}

\xiaoti{他 \kh[1cm] 卖房度日,\kh[1cm] 决不在日本侵略者的统治下登台演出。}

\xiaoti{马铃薯和藕 \kh[1cm] 植物的根, \kh[1cm] 茎。}

\xiaoti{\kh[1cm] 鹅的眼睛看人,觉得人比鹅小,\kh[1cm] 鹅不怕人。}

\xiaoti{\kh[1cm] 戏园子老板开出的条件多么优厚,梅兰芳 \kh[1cm] 拒绝了。}

\zhongti{补充词语并按要求答题。}

人声 \kh 沸 \kbai 锣鼓 \kh 天  \kbai  \underline{三头六 \kh}   \kbai 震耳 \kh 聋

 \kh 声细语  \kbai \CJKunderwave{秉公 \kh 法}  \kbai  鸦 \kh 无声  \kbai \kh 无声息

\xiaoti{根据意思选词语。}

\xiaoxiaoti{连乌鸦和麻雀的声音都没有,形容非常安静。\kh[2cm]}

\xiaoxiaoti{耳朵都快震聋了,形容声音很大。 \kh[2cm]}

\xiaoti{画“\xhx[1cm]”的词语含有数字,这样的词语还有 \xhx 、 \xhx 。 画 “\CJKunderwave{\hspace{1cm}} ” 的词语是描写人物品质的,这样的词语还有  \xhx 、 \xhx 。}

\xiaoti{词语“\kh 声细语"的近义词是 \xhx 。}

\xiaoti{选择词语填入句子中。}

\xiaoxiaoti{在大桥通车典礼上,彩旗飘扬,\kh[2cm] ,\kh[2cm] ,热闹非凡。}

\xiaoxiaoti{秋天 \kh[2cm] 地来了,迈着轻盈的步子,带着收获的希望和喜悦。}

\dati{易错题精选。}

\zhongti{写出下列词语的近义词和反义词。}

\xiaoti{写近义词。}

\kbai 猛烈 —— \kh[1.5cm] \kbai 绝望 —— \kh[1.5cm] \kbai  期待 —— \kh[1.5cm]

\newpage

\kbai 果真 —— \kh[1.5cm] \kbai 尤其 —— \kh[1.5cm] \kbai  指望 —— \kh[1.5cm]

\xiaoti{写反义词。}

\kbai 舒适 —— \kh[1.5cm] \kbai 僵硬 —— \kh[1.5cm] \kbai  均匀 —— \kh[1.5cm]

\kbai 庞大 —— \kh[1.5cm] \kbai 自豪 —— \kh[1.5cm] \kbai  训斥 —— \kh[1.5cm]

\zhongti{根据意思写四字词语}

\xiaoti{使刮风下雨,原指神仙道士的法力,现在比喻能够支配自然或左右某种局面。\kh[2cm]}

\xiaoti{传说中指利用法术乘云雾飞行。形容奔驰迅速或头脑迷糊,感到身子轻飘飘的。\kh[2cm]}

\xiaoti{精神非常疲劳,体力消耗已尽,形容极度疲乏。\kh[2cm]}

\xiaoti{水流奔腾而不停止。\kh[2cm]}

\xiaoti{冰雪漫天盖地,形容非常寒冷。\kh[2cm]}

\xiaoti{好像在思考着什么似的。\kh[2cm]}

\xiaoti{心里怎么想,手就能怎么做,形容运用自如。\kh[2cm]}

\xiaoti{跟土的颜色一样,没有血色。形容极端惊恐。\kh[2cm]}

\zhongti{选词填空。}

\centerline{温和 \kbai 温暖 \kbai 暖和}

\xiaoti{春天的一个早晨,当母亲准备出去工作的时候,阳光 \kh[1.5cm] 地从那个小窗子射进来,一直射到地上。}

\xiaoti{“太阳今天在我身上照得怪 \kh[1.5cm]  的。”小女孩说,}

\xiaoti{她快乐地坐在 \kh[1.5cm]  的太阳光中。}

\centerline{改造 \kbai 改善 \kbai 改变}

\xiaoti{正是这些发现和发明,使人类的生活大大改观,其 \kh[1.5cm] 的程度超过了人类历史上百万年的总和。}

\xiaoti{在新的世纪里,现代科学技术必将继续创造一个个奇迹,不断 \kh[1.5cm]  我们的生活。}

\xiaoti{只有强者,才能排除万难, \kh[1.5cm]  世界,驾驭世界。}

\end{document}

