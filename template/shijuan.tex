% 绘制带圈的数字
\usepackage{tikz}
\usepackage{etoolbox}
\newcommand{\circled}[2][]{\tikz[baseline=(char.base)]
    {\node[shape = circle, draw, inner sep = 1pt]
    (char) {\phantom{\ifblank{#1}{#2}{#1}}};%
    \node at (char.center) {\makebox[0pt][c]{#2}};}}
\robustify{\circled}

% 题目相关的配置
\newcounter{itemdati}[section]          % 大题的计数器: 一、二、三、……
\newcounter{itemzhongti}[itemdati]      % 大题的计数器:(一)(二)(三)……
\newcounter{itemxiaoti}[itemzhongti]    % 小题的计数器: 1. 2. 3. ……
\newcounter{itemxiaoxiaoti}[itemxiaoti] % 小小题的计数器: (1) (2) (3) ……
\newcounter{itemweiti}[itemxiaoxiaoti]  % 微题的计数器: ①  ②  ③ 	 ……
\newcommand{\dati}[1]{ % 大题的标题
  \stepcounter{itemdati}
  \hspace{-2.8em}\textbf{\chinese{itemdati}、 #1}
}
\newcommand{\zhongti}[1]{ % 中题的标题
  \stepcounter{itemzhongti}
  \hspace{-2.8em}(\chinese{itemzhongti}) #1
}
\newcommand{\xiaoti}[1]{ % 小题的标题
  \stepcounter{itemxiaoti}
  \hangafter 1\setlength{\hangindent}{3.2em}{\arabic{itemxiaoti}. #1}
}
\newcommand{\xiaoxiaoti}[1]{ % 小小题的标题
  \stepcounter{itemxiaoxiaoti}
  \hangafter 1\setlength{\hangindent}{5.4em}{\hspace{1em}(\arabic{itemxiaoxiaoti})#1}
}
\newcommand{\weiti}[1]{ % 微题的标题
  \stepcounter{itemweiti}
  \hangafter 1\setlength{\hangindent}{5.4em}{\hspace{3em}\circled{\arabic{itemweiti}}#1}
}

\newcommand{\xhx}[1][6em] {\underline{\hspace{#1}}} % 下划线

\newcommand{\kh}[1][2em] {\mbox{(\hspace{#1})}} % 括号

\newcommand{\kbai}[1][1em] {\hspace{#1}} % 空白

% 带括号:指题目后面,有一个向右对齐的括号。形如:
%    题目标题  。。。。 (   )
% 常用于 判断对错 或 填写选择项
\newcommand{\daikuohao}[1]{#1 \hfill \kh}

%---- 将若干项目放置在一行 -----
% 这一组命令的 第1个参数(可选),用于控制每个项目所占的宽度
\newcommand{\twoInLine}  [3][10em] {\begin{tabular}[t]{*{2}{@{}p{#1}}} #2 & #3\end{tabular}}
\newcommand{\threeInLine}[4][10em] {\begin{tabular}[t]{*{3}{@{}p{#1}}} #2 & #3 & #4\end{tabular}}
\newcommand{\fourInLine} [5][10em] {\begin{tabular}[t]{*{4}{@{}p{#1}}} #2 & #3 & #4 & #5\end{tabular}}
\newcommand{\fiveInLine} [6][6.8em]{\begin{tabular}[t]{*{5}{@{}p{#1}}} #2 & #3 & #4 & #5 & #6\end{tabular}}
\newcommand{\sixInLine}  [7][5.8em]{\begin{tabular}[t]{*{6}{@{}p{#1}}} #2 & #3 & #4 & #5 & #6 & #7\end{tabular}}

%--- 将若干项目放置为一列(前加表示分组的大括号) ---
\newcommand{\twoInRow}  [2] {$\Bigg\{ \begin{tabular}{@{}l@{}} \text{#1}\\ \text{#2} \end{tabular} $ }
\newcommand{\threeInRow}[3] {$\Bigg\{ \begin{tabular}{@{}l@{}} \text{#1}\\ \text{#2}\\ \text{#3} \end{tabular} $ }
\newcommand{\fourInRow} [4] {$\Bigg\{ \begin{tabular}{@{}l@{}} \text{#1}\\ \text{#2}\\ \text{#3}\\ \text{#4} \end{tabular} $ }

%---- 选择题选项 ----
% 四个选项排成一行
\newcommand{\fourch}[4]{\begin{tabular}{*{4}{@{}p{3.5cm}}}(A)~#1 & (B)~#2 & (C)~#3 & (D)~#4\end{tabular}}
% 每两个选项排成一行,共两行
\newcommand{\twoch}[4]{\begin{tabular}{*{2}{@{}p{7cm}}}(A)~#1 & (B)~#2\end{tabular}\\\begin{tabular}{*{2}{@{}p{7cm}}}(C)~#3 &(D)~#4\end{tabular}}
% 每个选项单独占一行,共四行
\newcommand{\onech}[4]{(A)~#1 \\ (B)~#2  \\ (C)~#3  \\ (D)~#4}  % 一行
