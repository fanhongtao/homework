\documentclass[UTF8]{ctexart}
\usepackage{amsmath}
\usepackage{geometry}
\usepackage{CJKfntef}  % 文字下加点
\geometry{a4paper,left=2cm,right=2cm,top=2cm,bottom=2cm}
{\linespread{1.9}

% \pagestyle{plain}
% 绘制带圈的数字
\usepackage{tikz}
\usepackage{etoolbox}
\newcommand{\circled}[2][]{\tikz[baseline=(char.base)]
    {\node[shape = circle, draw, inner sep = 1pt]
    (char) {\phantom{\ifblank{#1}{#2}{#1}}};%
    \node at (char.center) {\makebox[0pt][c]{#2}};}}
\robustify{\circled}

% 题目相关的配置
\newcounter{itemdati}[section]          % 大题的计数器: 一、二、三、……
\newcounter{itemzhongti}[itemdati]      % 大题的计数器:(一)(二)(三)……
\newcounter{itemxiaoti}[itemzhongti]    % 小题的计数器: 1. 2. 3. ……
\newcounter{itemxiaoxiaoti}[itemxiaoti] % 小小题的计数器: (1) (2) (3) ……
\newcounter{itemweiti}[itemxiaoxiaoti]  % 微题的计数器: ①  ②  ③ 	 ……
\newcommand{\dati}[1]{ % 大题的标题
  \stepcounter{itemdati}
  \hspace{-2.8em}\textbf{\chinese{itemdati}、 #1}
}
\newcommand{\zhongti}[1]{ % 中题的标题
  \stepcounter{itemzhongti}
  \hspace{-2.8em}(\chinese{itemzhongti}) #1
}
\newcommand{\xiaoti}[1]{ % 小题的标题
  \stepcounter{itemxiaoti}
  \hangafter 1\setlength{\hangindent}{3.2em}{\arabic{itemxiaoti}. #1}
}
\newcommand{\xiaoxiaoti}[1]{ % 小小题的标题
  \stepcounter{itemxiaoxiaoti}
  \hangafter 1\setlength{\hangindent}{5.4em}{\hspace{1em}(\arabic{itemxiaoxiaoti})#1}
}
\newcommand{\weiti}[1]{ % 微题的标题
  \stepcounter{itemweiti}
  \hangafter 1\setlength{\hangindent}{5.4em}{\hspace{3em}\circled{\arabic{itemweiti}}#1}
}

\newcommand{\xhx}[1][6em] {\underline{\hspace{#1}}} % 下划线

\newcommand{\kh}[1][2em] {\mbox{(\hspace{#1})}} % 括号

\newcommand{\kbai}[1][1em] {\hspace{#1}} % 空白

% 选择题题目
\newcommand{\xuanzeti}[1]{#1 \hfill \kh}

%---- 将若干项目放置在一行 -----
% 这一组命令的 第1个参数(可选),用于控制每个项目所占的宽度
\newcommand{\twoInLine}  [3][10em] {\begin{tabular}[t]{*{2}{@{}p{#1}}} #2 & #3\end{tabular}}
\newcommand{\threeInLine}[4][10em] {\begin{tabular}[t]{*{3}{@{}p{#1}}} #2 & #3 & #4\end{tabular}}
\newcommand{\fourInLine} [5][10em] {\begin{tabular}[t]{*{4}{@{}p{#1}}} #2 & #3 & #4 & #5\end{tabular}}
\newcommand{\fiveInLine} [6][6.8em]{\begin{tabular}[t]{*{5}{@{}p{#1}}} #2 & #3 & #4 & #5 & #6\end{tabular}}
\newcommand{\sixInLine}  [7][5.8em]{\begin{tabular}[t]{*{6}{@{}p{#1}}} #2 & #3 & #4 & #5 & #6 & #7\end{tabular}}

%---- 选择题选项 ----
% 四个选项排成一行
\newcommand{\fourch}[4]{\begin{tabular}{*{4}{@{}p{3.5cm}}}(A)~#1 & (B)~#2 & (C)~#3 & (D)~#4\end{tabular}}
% 每两个选项排成一行,共两行
\newcommand{\twoch}[4]{\begin{tabular}{*{2}{@{}p{7cm}}}(A)~#1 & (B)~#2\end{tabular}\\\begin{tabular}{*{2}{@{}p{7cm}}}(C)~#3 &(D)~#4\end{tabular}}
% 每个选项单独占一行,共四行
\newcommand{\onech}[4]{(A)~#1 \\ (B)~#2  \\ (C)~#3  \\ (D)~#4}  % 一行


% 设置页脚
\usepackage{fancyhdr}
\pagestyle{fancy}
\renewcommand\headrulewidth{0pt}
\fancyfoot[C]{四年级(上) 语文(R) \kbai 13 — \thepage}

\usepackage{tabularx}

\begin{document}

\begin{center}
\begin{LARGE}
\textbf{四年级(上) 语文(R)}
\end{LARGE}

\begin{Large}
期末专项复习卷(一)

拼音、字词
\end{Large}

\begin{large} 班 级\xhx \kbai 姓 名\xhx \end{large}
\end{center}

\dati{基础题精选。}

\xiaoti {把下列生字的音节补充完整。}

\newcommand{\xx}{\xhx[2em]}

\begin{tabularx}{42em}{*{7}{>{\centering\arraybackslash}X}}
    \xx ùn & \xx uàn & \xx ū & \xx ái & \xx ì  & \xx áng  & \xx ǎo \\
    顿 & 贯 & 输 & 宅 & 避 & 囊 & 扰 \\
    x \xx & k \xx  & q \xx & j \xx & t \xx & b \xx & k \xx \\
    蓄 & 溃 & 签 & 竞 & 淌 & 拜 & 炕
\end{tabularx}

\xiaoti{给带点字选择正确的读音,打“$\surd$”。\\
  \begin{tabular}[t]{*{4}{p{9em}}}
    \CJKunderdot{屹}立 (qí \kbai yì)   & \CJKunderdot{霎}时 (sà \kbai shà) &  稻\CJKunderdot{穗} (suì \kbai huì) &  田\CJKunderdot{埂} (gěng \kbai gěn)\\
    \CJKunderdot{囚}犯 (quí \kbai qiú) & \CJKunderdot{启}示 (qí \kbai qǐ) &  \CJKunderdot{潜}入 (qiān \kbai qián) & \CJKunderdot{均}匀 (jǖn \kbai jūn)\\
    \CJKunderdot{骤}雨 (jǜ \kbai zhòu) & \CJKunderdot{抛}出 (pāo \kbai bāo) &  \CJKunderdot{少}女 (shǎo \kbai shào) & 饶\CJKunderdot{恕} (shù \kbai nù)\\
    \CJKunderdot{拯}救 (zhěn \kbai zhěng) & \CJKunderdot{笋}芽 (sǔn \kbai yǐn) &  开\CJKunderdot{凿} (zāo \kbai záo) & 功\CJKunderdot{绩} (jī \kbai jì)\\
  \end{tabular}
}

\newcommand{\duoyinzi}[3]{#1 \twoInRow{#2 \kh[4em]}{#3 \kh[4em]}}
\xiaoti{多音字组词。\\
\threeInLine[13em]{\duoyinzi{系}{xì}{jì}}
    {\duoyinzi{降}{xiáng}{jiàng}}
    {\duoyinzi{曲}{qū}{qǔ}}\\
\threeInLine[13em]{\duoyinzi{露}{lòu}{lù}}
    {\duoyinzi{角}{jué}{jiǎo}}
    {\duoyinzi{哄}{hōng}{hòng}}\\
\threeInLine[13em]{\duoyinzi{还}{huán}{hái}}
    {\duoyinzi{宁}{nìng}{níng}}
    {\duoyinzi{要}{yāo}{qǔ}}
}

\xiaoti{给带点的多音字选择正确的读音。}

\centerline{jǐ \kbai jì}

\xiaoxiaoti{今天我讲话的主题是“遵守学校\CJKunderdot{纪} \kh 律,共建文明校园”。}

\xiaoxiaoti{《\CJKunderdot{纪} \kh 昌学射》这则寓言故事告诉我们,只要专心致志、坚持不懈去下一番苦功夫,就一定能取得较大的成绩。}

\centerline{xuán \kbai xuàn}

\xiaoxiaoti{无论嵌上滚珠,还是钉上铁钉,冰尜儿都不会裂开,能毫无怨言地让你抽打,在冰面上\CJKunderdot{旋} \kh 转,舞蹈。}

% \newpage %--------------

\xiaoxiaoti{李逵是小说《水浒传》中的主要人物,他肤色黑幼黑,性格暴烈,心粗胆大,绰号“黑\CJKunderdot{旋} \kh 风”。}

\xiaoti{看拼音写词语。}

\newcommand{\khao}{\kh[1.5cm]}

\begin{tabularx}{42em}%
{*{5}{>{\centering\arraybackslash}X}}
gǔn dòng & luǎn shí & jiāng yìng & xū ruò\\
\khao &  \khao & \khao & \khao \\

jí shǐ & ào mì & bēi cǎn & měng shòu \\
\khao &  \khao & \khao & \khao \\

shēn qū & bó dòu & páng dà & jià rì \\
\khao &  \khao & \khao & \khao \\

fā chàn & wán pí & zhù wēi & jǐn zhāng \\
\khao &  \khao & \khao & \khao \\

chè huàn & zá guō & kuì bài & yán sù \\
\khao &  \khao & \khao & \khao \\

xiōng huái & yí huò & xùn chì & qiú ráo \\
\khao &  \khao & \khao & \khao \\
\end{tabularx}

\xiaoti{给下列加点字选择正确的解释,填在句子后面的括号里。\\
    观:A.看;B.看到的景象;C.对事物的认识,看法。
}

\xiaoxiaoti{钱塘江大潮,自古以来被称为天下奇\CJKunderdot{观}。\kh }

\xiaoxiaoti{我们的班主任老师教学\CJKunderdot{观}念非常先进。\kh }

\xiaoxiaoti{去年暑假,我和爸爸、妈妈一起参\CJKunderdot{观}了故宫博物院。\kh}

\dati{较难、易错题精选。}

\xiaoti{下列每组词语中都有一个加点字的读音与其他两个不同,把它选出来。(填序号)}

\newcommand{\threeChoice}[3]{\threeInLine{A. #1}{B. #2}{C. #3}}

\xiaoxiaoti{\threeChoice{\CJKunderdot{荧}屏}{苍\CJKunderdot{蝇}}{\CJKunderdot{隐}蔽} \kh}

\xiaoxiaoti{\threeChoice{\CJKunderdot{即}使}{\CJKunderdot{既}然}{\CJKunderdot{技}术} \kh}

\xiaoxiaoti{\threeChoice{\CJKunderdot{媒}体}{明\CJKunderdot{媚}}{倒\CJKunderdot{霉}} \kh}

\xiaoxiaoti{\threeChoice{\CJKunderdot{证}明}{出\CJKunderdot{征}}{竞\CJKunderdot{争}} \kh}

\xiaoti{读句子,选择正确的读音,用“ \xhx[2em] ”标出。}

\xiaoxiaoti{盘古发出的声音化作了\CJKunderdot{隆}(lōng \kbai lóng)隆的雷声。}

\xiaoxiaoti{在鲫鱼背前,爸爸给我和老爷爷照了一张\CJKunderdot{相}(xiāng \kbai xiàng),留作纪念。}

\xiaoxiaoti{顶不济的,也要\CJKunderdot{钉}(dīng \kbai dìng)上一枚铁\CJKunderdot{钉}(dīng \kbai dìng),否则转不了多少圈,尖部就会开裂。}

\xiaoxiaoti{细丝原先是直的,现在弯\CJKunderdot{曲}(qū \kbai qǔ)了,把爬山虎的嫩茎拉一把,使它紧贴在墙上。}

% \newpage %--------------

\xiaoti{读拼音,写字词。}

\xiaoxiaoti{秋天到,秋天到,田里zhuāng jia \khao 长得好。棉花朵朵白,wān dòu \khao 粒粒饱。高粱涨红了脸,dào zi \khao 笑弯了腰。秋天到,秋天到,园里果子长得好。枝头结柿子,架上挂 pú tao \khao 。}

\xiaoxiaoti{又不知过了多少年,天和地终于成形了,盘古也 lèi \khao 得倒下了。他的 xuè yè \khao 变成了 bēn liú bù xī \kh[8em] 的江河;他的汗毛变成了 mào shèng \khao 的花草树木;他的汗水变成了zī rùn \khao 万物的雨露……}

\xiaoxiaoti{学校 cāo chǎng \khao 北边墙上满是 pá shān hǔ \kh[8em] 。它的 nèn yè \khao 绿得那么新鲜,看着非常舒服。叶尖一顺儿朝下,在墙上铺得那么 jūn yún \khao ,没有 chóng dié \khao 起来的,也不留一点儿 kòng xì \khao 。}

\xiaoti{查字典。\\
\-\hspace{2em}“家雀儿"的“雀"用音序查字法,应先查音序 \kh ,再查音节 \kh ;用部首查字法,应先查部首 \kh ,再查 \kh 画;它的另一个读音是 \kh ,可组词 \khao 。
}

\newcommand{\tongyizi}[4]{\makebox[5em][l]{#1}\threeInLine{#2}{#3}{#4}}

\xiaoti{巧填同音字。\\
  \tongyizi{dào}{水 \kh}{\kh 路}{来 \kh}\\
  \tongyizi{nì}{\kh 水}{\kh 风}{\kh 名} \\
  \tongyizi{chuí}{\kh 背}{\kh 直}{铁 \kh} \\
  \tongyizi{sù}{严 \kh}{快 \kh}{朴 \kh} \\
  \tongyizi{gài}{\kh 房}{灌 \kh}{大 \kh} \\
  \tongyizi{zhù}{\kh 军}{\kh 子}{\kh 宅}
}

\xiaoti{在括号里填上合适的 “huàn” 字。}

\xiaoxiaoti{20世纪是一个呼风 \kh 雨的世纪。}

\xiaoxiaoti{我们的祖先大概谁也没有料到,他们的那么多 \kh 想在现代纷纷变成了现实。}

\xiaoxiaoti{随他说去吧,我这个配角虽然配不上他,可是老师没撤 \kh 我,他也只好将就。}

\xiaoti{辨字组词。}

\newcommand{\diff}[2]{\twoInRow{#1 \kh}{#2 \kh}}

\sixInLine[8em]{\diff{掐}{陷}}
    {\diff{绳}{蝇}}
    {\diff{溉}{概}}
    {\diff{掩}{淹}}
    {\diff{溃}{馈}}
    {\diff{暮}{慕}}


\xiaoti{用横线画出下列词语中的错别字,并把正确的写在括号里。\\
  \threeInLine{起头并进\kh}{山崩地列\kh}{横七坚八\kh} \\
  \threeInLine{腾云架雾\kh}{喷喷不平\kh}{无可奈河\kh} \\
  \threeInLine{通情达礼\kh}{重整棋鼓\kh}{惹有所思\kh}
}

\xiaoti{按要求填空。\\
  \-\hspace{2em} “堂”字用音序查字法,应先查大写字母 \xhx ,它的读音是 \xhx 。
  用部首查字法,应先查部首 \xhx ,再查 \xhx 画。
  “堂”字在字里有四种解释: \circled{1} 正房,高大的屋子; 
  \circled{2} 指同宗但不是嫡亲的(亲属);
  \circled{3} 过去官吏审案办事的地方; 
  \circled{4} 量词。
  下列词语中的“堂”应选哪个解释呢?请把最恰当解释的序号填入后面的括号里。\\
  对簿公堂\kh \kbai 一堂优质课\kh \kbai 堂兄\kh \kbai 哄堂大笑\kh
}

\xiaoti{辨析下列多义字,把序号填在括号里。}

\-\hspace{3em} 探:\circled{1}寻求,探索。 \circled{2}侦察,暗中考察。\circled{3}探望,访问。 \circled{4} (头或上体)伸出。

\xiaoxiaoti{小女孩望着最低的那块窗玻璃说:“有个绿东西从窗玻璃旁边\CJKunderdot{探}出头来,它是什么呢?” \kh}

\xiaoxiaoti{天才在于创新,智者源于\CJKunderdot{探}索。\kh}

\xiaoxiaoti{解放军抓住了几个前来\CJKunderdot{探}听军情的特务。\kh}

\xiaoxiaoti{星期六,我去敬老院\CJKunderdot{探}望老爷爷老奶奶。\kh}

\-\hspace{3em}\hangafter 1\setlength{\hangindent}{5.4em} 熟:\circled{1}食物烧煮到可吃的程度。\circled{2}成熟,植物的果实或种子长成。\circled{3}程度深。\circled{4}因常接触而知道得清楚。

\xiaoxiaoti{看,稻谷就要成\CJKunderdot{熟}了,稻穗低垂着头,稻田像一块月光镀亮的银毯。\kh}

\xiaoxiaoti{放学后,妈妈已经把可口的饭菜做\CJKunderdot{熟}了。\kh}

\xiaoxiaoti{望着弟弟\CJKunderdot{熟}睡的样子,妈妈开心地笑了。\kh}

\xiaoxiaoti{公交车司机师傅对这条线路很\CJKunderdot{熟}悉。\kh}

\dati{拓展题。(读故事,猜汉字。)}

三国时期,丞相曹操命令工匠们造了一所花园。造成时,曹操前去观看,他没有夸奖和批评,就叫人取了一支笔在花园门上写了一个“活”字便走了。大家都不了解其中的含义。这时有一个叫杨修的人对工匠们说,“门”添活字,你们不知道是什么字吗?丞相这是嫌你们把花园门造得太大了。于是工匠们重新建造园门。完工后工匠们再请曹操去观看,曹操很喜欢。聪明的你,一定知道这是哪个字了,请你把它写出来与大家分享吧!

\xhx[15cm]


\end{document}
