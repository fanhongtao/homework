\documentclass[UTF8]{ctexart}
\usepackage{amsmath}
\usepackage{geometry}
\usepackage{CJKfntef}  % 文字下加点
\geometry{a4paper,left=2cm,right=2cm,top=2cm,bottom=2cm}
{\linespread{1.8}

% \pagestyle{plain}
\input{../../template/shijuan}

% 设置页脚
\usepackage{fancyhdr}
\pagestyle{fancy}
\renewcommand\headrulewidth{0pt}
\fancyfoot[C]{四年级(上) 语文(R) \kbai 15 — \thepage}

\usepackage{tabularx}

\begin{document}

\begin{center}
\begin{LARGE}
\textbf{四年级(上) 语文(R)}
\end{LARGE}

\begin{Large}
期末专项复习卷(三)

句 \kbai 子
\end{Large}

\begin{large} 班 级\xhx \kbai 姓 名\xhx \end{large}
\end{center}

\dati{基础题精选。}

\newcommand{\tianjuzi}[1]{#1 \\ \xhx[14cm]}

\zhongti{把下面的词语整理成一句通顺的话,并加上标点符号。}

\xiaoti{\tianjuzi{鹅卵石 \kbai 的 \kbai 河床 \kbai 灰白色 \kbai 布满}}

\xiaoti{\tianjuzi{太阳 \kbai 晒得 \kbai 把 \kbai 暖洋洋的 \kbai 在外边照着 \kbai 豆荚}}

\xiaoti{\tianjuzi{风 \kbai 摇撼着 \kbai 白桦树 \kbai 猛烈地 \kbai 路旁的}}

\xiaoti{\tianjuzi{爸爸 \kbai 带我 \kbai 去黄山 \kbai 爬天都峰 \kbai 假日里}}

\xiaoti{\tianjuzi{爬山虎 \kbai 往上爬 \kbai 一脚一脚地 \kbai 就是 \kbai 这样}}

\zhongti{按要求完成句子练习。}

\xiaoti{\tianjuzi{西门豹把巫婆和官绅的头子投进了漳河。(改为“被”字句)}}

\xiaoti{\tianjuzi{科学家把蝙蝠能在夜里飞行的秘密揭开了。(改为“被”字句)}}

\xiaoti{\tianjuzi{普罗米修斯摇摇头,坚定地回答:“为人类造福,有什么错?我可以忍受各种痛苦,但决不会承认错误,更不会归还火种!”(改为转述句)}}

\xiaoti{\tianjuzi{人不是生来就懂得许多道理的,谁能没有疑惑呢?(给句子换种说法,意思不变)}}

\xiaoti{\tianjuzi{我套上老虎皮,戴上老虎头罩,紧张地等候。(缩句)}}

\newpage % -----------

\xiaoti{\tianjuzi{用恰当的关联词将下面的两句话合为一句话。\\蟋蟀选择住址要排水优良。\kbai 蟋蟀选择住址要有温和的阳光。}}

\xiaoti{\tianjuzi{要是在路上碰到鹅,我们就绕个大圈子才敢走过去(改为双重否定句)}}

\xiaoti{\tianjuzi{当四周很安静的时候,蟋蟀就在这平台上弹琴。(仿写拟人句)}}

\xiaoti{\tianjuzi{是谁来呼风唤雨呢?当然是人类。(仿写设问句)}}

\xiaoti{\tianjuzi{\CJKunderdot{自从}有了火,人类\CJKunderdot{就}开始用它烧熟食物,驱寒取暖,并用火来驱赶危害人类安全的猛兽……(用加点的词语写句子)}}

\zhongti{根据意思写诗句。}

\xiaoti{\tianjuzi{一道残阳倒映在江面上,阳光照射下,波光粼粼,金光闪闪,一半呈现出深深的碧绿,一半呈现出红色。}}

\xiaoti{\tianjuzi{梅花须逊让雪花三分品莹洁白,雪花却输给梅花一段清香。}}

\xiaoti{\tianjuzi{落日的余晖返入深林,又照到林中的青苔上。}}

\xiaoti{\tianjuzi{如果喝醉了,倒在战场上,也请你不要笑话,自古以来出征打仗的人有几个能活着回来?}}

\zhongti{指出下面各句所用的修辞手法。}

\xiaoti{\daikuohao{浪潮越来越近,犹如千万匹白色战马齐头并进,浩浩荡荡地飞奔而来。}}

\xiaoti{\daikuohao{从果园那边飘来果子的甜香,是雪梨,是火把梨,还是紫葡萄?都有。}}

\xiaoti{\daikuohao{看,稻谷就要成熟了,稻穗低垂着头。}}

\xiaoti{\daikuohao{走过月光闪闪的溪岸,走过石拱桥,走过月影团团的果园,走过庄稼地和菜地……}}

\xiaoti{\daikuohao{它们是那样柔弱,怎么禁得起这猛烈的风雨呢?}}

\zhongti{\daikuohao{下列句子中标点符号使用不正确的一项是}\\
  \onech{多么奇妙的夜晚啊,我和阿妈走月亮!}
    {最小的一粒豌豆说:“是的,事情马上就要揭晓了。”}
    {难道蝙蝠的眼睛特别敏锐?能在漆黑的夜里看清楚所有的东西吗?}
    {20世纪的成就,真可以用“忽如一夜春风来,千树万树梨花开”来形容。}
}

\dati{较难、易错题精选。}

\zhongti{选择题。}

\xiaoti{\daikuohao{下面句子中的关联词语使用正确的一项是} \\
  \onech{蟋蟀的出名即使由于它的唱歌,也由于它的住宅。}
    {一旦病发展到肠胃里,服几剂汤药也还能治好。}
    {虽然你已经取得了不小的成績,但你的眼力还不够。}
    {如果戏园子老板开出的条件多么优厚、梅兰芳就护绝了。}
}

\xiaoti{\daikuohao{下列句子与例句意思不同的一项是} \\
  花朵自己已经被雨点打得抖个不停了,怎能容它们藏身呢?\\
  \onech{花朵自己已经被雨点打得抖个不停了,不能容它们藏身。}
    {花朵自己即使被雨点打得抖个不停了,还能容它们藏身。}
    {花朵自己已经被雨点打得抖个不停了,无法再容它们藏身。}
    {花朵自己已经被雨点打得抖个不停了,又如何能容它们藏身呢?}
}

\zhongti{判断下面句子所运用的修辞手法,把序号填在括号里。}

\centerline{A.比喻 \kbai B.拟人 \kbai C.设问 \kbai D.排比 \kbai E.反问 }

\xiaoti{\daikuohao{每个小水塘都抱着一个月亮!}}

\xiaoti{\daikuohao{月光是那样柔和,照亮了高高的点苍山,照亮了村头的大青树,也照亮了,照亮了村间的大道和小路……}}

\xiaoti{\daikuohao{靠什么呼风唤雨呢?靠的是现代科学技术。}}

\xiaoti{\daikuohao{可怜九月初三夜,露似真珠月似弓。}}

\xiaoti{\daikuohao{这是在外国人的地盘里,谁又敢怎么样呢?}}

\zhongti{修改病句。}

\xiaoti{\tianjuzi{五颜六色的红旗在空中迎风飘动。}}

\xiaoti{\tianjuzi{听了金奎叔的话,使我受到了很大的教育。}}

\xiaoti{\tianjuzi{千千万万数不清的小学生参加了“关爱生命,预防溺水”征文比赛。}}

\newpage %---------------

\xiaoti{\tianjuzi{我们不但要改造世界,而且要认识世界。}}

\xiaoti{\tianjuzi{王阿姨从超市买回了芹菜、菠菜、莴苣、菠萝、茄子……等蔬菜。}}

\zhongti{综合改错(在原文上修改。)}

你看过“昆虫记”这本书吗?这是 19 世纪法国昆虫学家达尔文撰写的一本书。书里记裁了许多有趣的故事,它详细描述了昆虫生活、捕食、逃避敌害等知识,字里行间洋溢着作者对生命的敬爱。这本书因为是一部研究昆虫的科学巨著,所以是一部讴歌生命的宏伟诗篇。

\zhongti{用下列词语造句。}

\xiaoti{占上风: \xhx[12cm]}

\xiaoti{栽跟头: \xhx[12cm]}

\zhongti{照例子,仿写或续写句子。}

\xiaoti{(1)\tianjuzi{蝙蝠\CJKunderdot{一边}飞,\CJKunderdot{一边}从嘴里发出一种声音。(用上加点的词语造句)}\\
  (2)写害怕:我们马上都不说话了,贴着墙壁,悄悄地走过去。\\
  写伤心:\xhx[11.5cm]\\
  写\kh : \xhx[11cm]
}

\xiaoti{例:\underline{母爱是一缕阳光,让你的心灵即使在寒冷的冬天感到温暖如春。}\\
  父爱 \xhx[12cm]。
}

\xiaoti{例:\underline{事业说:人生就是建筑历史的一块砖瓦;} \\
  奋斗说:\xhx[11.5cm]; \\
  希望说:\xhx[11.5cm]; \\
  友谊说:\xhx[11.5cm]。
}

\zhongti{按一定顺序给下列句子排序。}

\kh 春天到了,小豌豆长出了小叶子,小姑娘看到后觉得好了一些。

\kh 一个星期后,这棵豌豆猛劲生长,小姑娘看到后很高兴,并能坐一个钟头的时间。

\kh 母亲从窗台上牵了一根绳子到窗框的上端去,使这棵豌豆苗可以顺着它向上生长。

\kh 这棵豌豆开出了紫色的豌豆花,小姑娘低下头轻轻地吻了一下她柔嫩的叶子。

\kh 在这个小小的顶楼里住着一个穷苦的女人,她有一个病得很厉害的女儿。

\hangafter 1\setlength{\hangindent}{5.4em}\kh 最后一粒豆子被射到空中,落到顶楼窗子下面的一块旧板子上,正好钻进一个长满了青苔的裂缝里。

\end{document}
