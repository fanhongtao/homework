\documentclass[UTF8,10pt]{ctexart}
\usepackage{amsmath}
\usepackage{geometry}
\geometry{a4paper,left=2cm,right=2cm,top=2cm,bottom=2cm}

\input{../../template/shijuan}

\newcommand{\sanxuanyi}[3]{\threeInLine{A. #1}{B. #2}{C. #3} }

  

\begin{document}

\dati{判断各组单词画线部分的读音是否相同,相同的写“S”,不同的写“D”}

\kh \xiaoti{o\underline{f}\kbai \underline{f}ly}

\dati{单项选择}

\xiaoti{-- Is there a teacher's office in the building?\\
  -- \xhx \\
  \sanxuanyi{Yes, there are}{No, there is}{Yes, there is}
}

\xiaoti{There \xhx some water in the bottle.\\
  \sanxuanyi{are}{is}{aren't}
}

\xiaoti{--Where is Su Nan?\\
  -- He is \xhx the playground.\\
  \sanxuanyi{on}{in}{at}
}

\xiaoti{Who \xhx you English? \\
  \sanxuanyi{teach}{teacher}{teaches}
}

\dati{选词填空。}

\xiaoti{How many \xhx (sandwitch, sandwitches) do you have?}

\xiaoti{Here \xhx (is, are) some tea \xhx (for, to) you.}

\xiaoti{Do you have any \xhx (drink, drinks)?}


\dati{按要求完成句子。}

\xiaoti{I'd like some ice creams.(改为一般疑问句)}

\xhx you like \xhx ice creams?

\xiaoti{I'd like a hot dog.(改为同义句)}

\xhx \xhx a hot dog.

\xiaoti{I have \underline{a} glass of apple juice.(对画线部分提问)}

\xhx \xhx \xhx of apple juice do you have?

\dati{根据上下文完成对话}

\xiaoti{
A: Good afternoon, boys. Can I \xhx you? \\
B: \xhx. A cup of \xhx, please. \\
A: The \xhx tea \xhx the black tea? \\
B: The green tea. What \xhx you, Mike? \\
C: I \xhx \xhx like tea. \xhx like a glass of apple juice. \\
A: Anything else? \\
B: No, \xhx . \\
A: \xhx you are. \\
B \& C: Thank you.
}

\dati{连词成句。}

\xiaoti{are, nice, they, all, very (.)}

\xhx[14cm]

\xiaoti{bar, is, a, here, snack (.)}

\xhx[14cm]

\xiaoti{coffee, come, of, a, drink, cup, and (.)}

\xhx[14cm]

\end{document}